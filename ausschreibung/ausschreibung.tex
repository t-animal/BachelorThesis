\documentclass[12pt,a4paper]{article}
\usepackage{german}
\usepackage{times}
\usepackage{hyperref}
\usepackage{xspace}
\usepackage{microtype}
%\usepackage{doublespace}
%------------------------------------------------------------------------------
%\setstretch{1.0}
\voffset-5mm
\hoffset-5mm
\textwidth17cm
\textheight24cm
\headsep0mm
\headheight0mm
\oddsidemargin0.3mm
\pagestyle{empty}
\parindent0mm
\parskip1ex
%------------------------------------------------------------------------------
%==============================================================================

\providecommand{\etal}[1]{#1\emph{~et~al.\xspace}}
\renewcommand\refname{References}

\begin{document}


\begin{center}
Bachelor Thesis at the Pattern Recognition Lab, FAU Erlangen-Nuremberg \hfill Nr.: ????\\[5mm]

\mbox{}\\
{\Large Android app for abstract recording of Go matches}
\end{center}

The game Go has multiple properties that make recording it interesting.
It is often played across country borders and even non-simultaneously (similar to correspondence chess). Even though a
notation similar to the one used in chess would be possible, it is not costumary. Thus one has to resort to digital
boards, e.g. on websites. An interface to fast recognize and transfer a player's move facilitates the use of real
pieces and at the same time allows for simple transfer of moves over large distances.

To improve one's play it is furthermore useful to record games and analyze them afterwards. Flaws in one's style of play
can be found this way and even games replayed from a specific state, in order to assess the impact of a particular
mistake. This is especially useful for unexperienced players.

This work should:
\begin{itemize}
	\item recognize a Go-Board  as soon as it is fairly in the center of the camera vision of an Android phone
	\item detect the current game state
	\item detect new moves
	\item save the game state in a compact form abstracting from the video
	\item consider robustness against shaking and movement of the camera
	\item tolerate at least part time occlusion of the board, because at least while putting down a piece or removing
	 some the board is inevitably occluded
\end{itemize}

Several methods of detection should be explored, especially regarding detection of the Go-Board (hough-lines and
LSD\cite{von2010lsd} or EDLines\cite{akinlar2011edlines} detection should be tested, as well as corner detection where
\cite{rosten2010faster} could be a good starting point). The effect of pre-processing steps such as denoising and
perspective correction should be explored (similar to \cite{jagannathan2005perspective}).

To evaluate the methods a data set has to be created. The algorithms should be evaluated against detection speed (fps)
and quality.

\begin{tabular}{ll}
\emph{Supervisors:} & Dipl.\,Inf.~V.~Christlein, Prof.\,Dr.\,-Ing.\ Joachim Hornegger\\
\emph{Student:} & Tilman Adler \\
\emph{Start:} & January 1st, 2015\\
\emph{End:} & June 1st, 2015\\
\end{tabular}
\nopagebreak[4]
\small
\bibliographystyle{unsrt}
\bibliography{ausschreibung}

\end{document}
%==============================================================================
