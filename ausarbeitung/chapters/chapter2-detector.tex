%!TEX root = ../Thesis.tex

\chapter{Detecting the board}
	When looking at a Go board what first catches one's eye are the lines. As we said before, though, relying solely on lines for the detection of the grid will probably fail in endgame situations. Therefore our algorithm uses them only to find intersections and tries to fill in gaps by adding information about pieces, too. This was born out of the idea that what is actually interesting are not the lines, but their intersections and a piece will always lie reasonably close to one. We also rely on the user placing the camera or the board such that the center of the board is located in the center of the screen.

	Our implementation was based on the OpenCV framework in its current version 2.4.10 and where not noted otherwise all detection has been performed on x86\_64 architecture. As described in chapter 3 this does not seem to provide different results than execution on mobile devices (i.e. ARM architecture), for example because of differences in floating point calculations.

	In short we do the following steps:
	\begin{enumerate}
		\item roughly pre-segment the board by analyzing connected components around the center of the thresholded input image
		\item detect horizontal and vertical lines and intersect them
		\item detect pieces on the board and consider their centers intersections, too
		\item remove duplicates
		\item select a few intersections around the center of the image
		\item build a submodel of the board by estimating where each selected point lies on the grid
		\item calculate their position in space using RANSAC and applying the resulting transformation matrix to a complete model of the board
	\end{enumerate}

	\section{Pre-processing}
	Mobile devices don't have the same computing power as desktop hardware does. Therefore our first step was to reduce the image size without losing information. To do so, we thresholded the grayscale input image using a mean adaptive threshold with a low constant value \emph{C} and a window size of what we expected to be roughly the width of one square on the board (approximately 45px, as measured in one of our sample images).

	Under the assumption that at least part of the board is in the center of the image we can segment it from the background now with high confidence by a simple connect-component analysis. On our testcase this failed only in one situation where the board lay in grass in the evening. Hard shadows connected the board with the background. On a flat surface this is not a problem.

	This step does not just improve speed but also detection performance because interfering background information like patterned wood table tops can be cut off.

	\section{Erkennen eines leeren Spielbretts}
		%Anwendung der Algorithmen
	\section{Erkennen eines Bretts mit Steinen}
	\section{Hochrechnen der Kreuzungen}
	\section{Erkennen der Farbe}
