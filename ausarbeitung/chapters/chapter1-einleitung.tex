%!TEX root = ../Thesis.tex

\chapter{Introduction}
	\section{Motivation}
	\label{introduction-motivation}
	The game of Go (Japanese \begingroup\setmainfont{Droid Sans Japanese}\small\begin{CJK}{UTF8}{min}囲碁\end{CJK}\endgroup ) is ancient and its ``origin [...] is lost in obscurity. [...] The traditional account that has won most credence ascribes the invention of Go to the [Chinese] Emperors Yau and Shun'' \cite{mihori1939japanese} (\textasciitilde 2300 BCE). But even though its rules are simpler than the rules of chess it is --- outside of Asia --- not nearly as wide-spread as the latter. When learning Go one comes to a certain point where one single error often decides the whole game. Often the winning player will be able to remove many of his or her opponent's pieces from the board, resulting in a devastating defeat. This can be quite frustrating. Other mistakes do not manifest until some moves after they have been made, which makes it hard to learn from them.

	The solution to this is obvious: Record a game and analyze it later, maybe resume it at the questionable move and see if it would have turned out otherwise. While there are advanced players who can remember an entire game after playing it especially new players have to rely on a notation with pen and paper, pocket computers or mobile apps. As this process drains on concentration many players tend to not note games.

	There are ways to save board games in an automated fashion, not requiring any interaction by the players. For chess there are specialized boards (DGT chess boards) starting at around 20€, which usually have special (magnetic) pieces \cite{bulsink2001device} or small holes with light sensors. They record a game and send the data to a computer, mobile device or other hardware for display and analysis. However for Go there is --- to our knowledge --- no such device on the consumer market. Also, the need for specialized hardware makes a solution like this very unappealing.

	What makes the situation worse is that while there is a notation called Kifu (Japanese \begingroup\setmainfont{Droid Sans Japanese}\small\begin{CJK}{UTF8}{min}棋譜\end{CJK}\endgroup ) for the recording of games it is --- again outside of Asia --- not very widespread. No numerical system as in chess (where the columns and rows are determined by a number and a letter) has been generally accepted. Ideally a recording system would save the moves in an independent fashion that makes it easy to transfer into the existing systems' formats. Another option would be to choose one system and save it in its data format. Once such a system exists there are other applications that can be thought of. For example playing against a computer or via the internet another human and using real pieces in the process. Anyhow it might lead to more players successfully learning how to play Go (well) and thus contribute to its spread in the western world, if just a little.

	The ubiquity of mobile devices with built-in cameras lets one solution appear very attractive: Have a mobile application record the game, analyze it via a computer vision application and save the result when it detects a new move. This	work intends to provide such a solution and show the efforts taken to create it.

	After presenting related work and possibly related patents we will first describe the part of the application that does the actual detection and go on to the integration of it in an Android app. Thereafter we will discuss the impact of different algorithms that have been evaluated regarding detection quality and speed. We also provide an assessment to what extent the goal of an automated recording device for Go games has been reached. We will conclude with an outlook on possibilities for future work improving upon the solution at hand and a summary of this work.


	\section{Related work}
	\label{introduction-work}
	The detection and augmenting of board games in general is illustrated by Eray Molla and Vincent Lepetit \cite{molla2010augmented}, who demonstrate the successful tracking of colored pawns on a Monopoly™ board.

	More closely related is  Steven Scher's, Ryan Crabb's and Jamie Davis's work \cite{scher2008making} who collect samples of a Go-board using a fixed camera in regular intervals. They detect the grid in the first image using \emph{Hough transformation} and assume it to be fixed. Afterwards they classify intersections (as ``lack'', ``white'' or ``empty'') by template images and use a Markov model to remove errors in the single images. To this, they model the detected state of every image and transitions in between according to the game rules.

	This approach yields high accuracy when capturing whole games and performing the detection asynchronously. On a mobile device this is not ideal as it drains the battery at a time the user does not expect it to. Also movement of the camera or the board is not supported at all as detection of the grid is only performed on the first image. However smart phones as a multi purpose device should be relocatable at least within a certain range. The user, for example, might have to take a call and then replace the phone. Thereafter the grid might not be detectable as easy any longer.
	\\

	Similarly Teemu Hirsim"aki \cite{hirsimaki2005extracting} uses the \emph{Hough space} to find parallel lines near the center of the image. He will then use this information in conjunction with the original image (filtered for edges) to build the board grid by solving a minimization problem on them. Intersections are classified by relating the median brightness to the median value in a certain window.

	The author assumes that the approach can be used on consecutive frames, but one problem remains when using only line detection: during later phases of the game many pieces will be placed especially in the center of the image. The lines are occluded by tokens and there is no way left to detect the board.
	\\

	Alexander K. Seewald \cite{seewald2010automatic} solved this by detecting the intersections using the \emph{SIFT algorithm}. His approach specifically targeted endgame situations. The keypoints are then classified (as before but additionally as ``empty inside the grid ┼'', ``empty on the edge ┬'' and ``empty on the corners ┐'') by an SVM. The results are then further refined by estimating the board position using the corners and classifying the previously missed tokens by average brightness.

	While he solved the problem of detecting end game states this came at the price of performance. With five seconds runtime on a PC for an 8 by 8 grid, this solution seems unpractical for slower mobile processors.

	\section{Related patents}
	\label{introduction-patents}

	There are several patents which protect apparatuses for the detection of Go games mostly from Japanese inventors. Most of them rely on some hardware device to detect where a piece has been placed. There are numerous approaches in this direction, for example using magnetic pieces \cite{JPH11114129A} or pressure sensitive mats \cite{JP2002000792A}.

	There are also some patents for work including a camera to record the game. A device from 2004, for example, uses a CMOS camera connected to a PC \cite{JP2004160170}. The camera is mounted on a pole next to the board which is permanently connected with the board, i.e. the camera angle is fixed. Another apparatus also uses a permanently fixed camera connected to a table \cite{JPH04307077A}. Here, the player can even play against a computer controlled robot arm. All those approaches use proprietary hardware and fixed setups, though.
