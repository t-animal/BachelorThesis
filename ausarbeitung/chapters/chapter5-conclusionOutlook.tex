%!TEX root = ../Thesis.tex

\chapter{Conclusion and outlook}
\section{Conclusion}
We could show that it is possible to detect a Go board, extract the current game state and record a game with the limited resources of a mobile device. Our approach uses results of line detection to detect the grid and from it intersections. This yields high rates of detected intersections on empty to sparsely occupied boards. To improve our detector we also add the center of roughly detected pieces to the set of intersections. From this we calculate the complete set of intersections starting from the center of board, which has to be provided by the user. Finally we evaluate the image at their locations to classify each as black, white or empty.

The quality of our detector is comparable to other work in this field. It can outperform these when there is a large number of pieces on the board. This allows us to start the detection mid-game or to stop it and continue it later and permits some movement of the camera. We can show that under extreme circumstances our application can detect the current game state. This goes so far that even boards which are completely covered in pieces can be classified.

Speedwise we can reach near real time performance on desktop hardware. On weaker mobile processors we reach about one frame per second, which beats previously existing algorithms.

\section{Outlook}
Due to the lack of a larger board we could not test our algorithm on more than a 9-by-9-grid. We are confident that our approach can be applied to other sizes, too, with minimal adjustments.

Another area of work could be to automatically detect the center of the board. This would remove any need for user interaction and allow the application to be run completely autonomously.

There are still some issues with specular highlights, which could also be improved upon. If they could be painted in well and fast, it could improve piece detection as well as color detection.
