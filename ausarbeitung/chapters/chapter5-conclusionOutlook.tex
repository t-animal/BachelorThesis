%!TEX root = ../Thesis.tex

\chapter{Conclusion and outlook}
\section{Conclusion}
We could show that it is possible to detect a Go board, extract the current game state and record a game with the limited resources of a mobile device. Our approach uses results of line detection to detect the grid and intersections. This yields high rates of detected intersections on empty to sparsely occupied boards. To improve our detector, we also add the center of roughly detected pieces to the set of intersections. From this we calculate the complete set of intersections starting from the center of board, which has to be provided by the user. Finally, we evaluate the image at their locations to classify each as black, white or empty.

The quality of our detector is comparable to other works in this field. It even outperforms them when there is a large number of pieces on the board. This allows us to start the detection mid-game or to stop it and continue it later and permits some movement of the camera. Our experiments show that we can detec the current game state under extreme circumstances. Even boards which are completely covered in pieces can be classified correctly.

Speed wise we can reach near real time performance on desktop hardware. On weaker mobile processors we reach about one frame per second, which beats previously existing algorithms.

Our contribution is a faster and reliable method for the detection of Go boards, which can handle densely occupied boards, too. We also showcased a possible end user product by embedding the developed detector in an android app.

\section{Outlook}
Due to the lack of a larger board we could not test our algorithm on more than a 9-by-9-grid. We are confident that our approach can be applied to other sizes, too, with minimal adjustments.

Another area of work could be to automatically detect the center of the board. This would remove any need for user interaction and allow the application to be run completely autonomously.

There are still some issues with specular highlights, which could also be improved upon. If they could be painted in well and fast, it could improve piece detection as well as color detection. Another option would be to improve the color detection step, because it is more prone to errors from specular highlights. One possibility could be to switch from rectangular windows to some more sophisticated shape like circles or discs. Obviously this would not improve piece detection, but both inpainting and different evaluation shapes might be combined.

Even though our detector's quality is comparable to the one in Scher's, Crab's and Davis's paper \cite{scher2008making} combining both could remove persistent errors. The modeling and path finding step from their approach could be added to our detector, especially as it was developed for a more general use. It should not take too much time and could be performed right after the detection was stopped.

Smartphones offer a multitude of additional sensory information, which might be used to improve the detection quality. For example, if the camera height above the board is known, one could calculate a transformation matrix as in \autoref{evaluation-prepostprocessing-perspectiveRectifying} based on the tilt of the phone. This can be determined using the built in accelerometers or gyroscopes.
